\documentclass{beamer}

% \usetheme{Warsaw}
% \usetheme{CambridgeUS}
% \usetheme{Berkeley}
% \usetheme{Antibes}
% \usetheme{Madrid}

\usetheme{Boadilla}


\begin{document}

\title{Object Detection}
\author{Peng Zhou}
\maketitle{}

\begin{frame}
  \frametitle{The Traditional Method}
  \begin{enumerate}

  \item Area selection
    \begin{itemize}
    \item The sliding window strategy is used to traverse the whole image. 
    \item The sliding windows have different scale, different aspect ratio.
    \item Too many redundant windows, high time complexity
    \end{itemize}

  \item Feature extraction
    \begin{itemize}
    \item SIFT, HOG
    \end{itemize}

  \item Classification
    \begin{itemize}
    \item SVM, Adaboost
    \end{itemize}
  \end{enumerate}
\end{frame}

\begin{frame}
  \frametitle{The Traditional Method}
  \emph{Two problems}
  \begin{enumerate}
  \item Sliding window selection is time complexity and redundant.
  \item The features of manual design are not very robust to the variation of diversity.
  \end{enumerate}
\end{frame}

\begin{frame}
  \frametitle{Deep Learning Method}
\begin{block}{}
  \begin{enumerate}
  \item Deep Learning Object Detection Algorithm Based on Region Proposal.
  \item Deep Learning Object Detection Algorithm Based on Regression.
  \item ...
  \end{enumerate}
\end{block}
\end{frame}

\begin{frame}
  \frametitle{DL based on Region Proposal}
  \href{https://people.eecs.berkeley.edu/~rbg/papers/pami/rcnn_pami.pdf}
  {Region-based Convolution Networks for Accurate Object detection and Segmentation}
  \begin{block}{Traditional}
    \begin{itemize}
    \item Sliding window + features of manual design 
    \end{itemize}
  \end{block}
  \begin{block}{R-CNN}
    \begin{itemize}
    \item Region proposal + CNN
    \end{itemize}
  \end{block}
  \begin{figure}[!t]
    \centering
    \includegraphics[width=0.7\textwidth]{pic/rcnn1.png}
    \caption{R-CNN}
  \end{figure}

\end{frame}

\begin{frame}
  \frametitle{R-CNN}
  
  
\end{frame}




\begin{frame}
  Add items
  \begin{itemize}
  \item \LaTeX{}
  \item beamer is easy.
  \end{itemize}

  \begin{enumerate}
  \item \LaTeX{}
  \item beamer is powerful
  \end{enumerate}
\end{frame}

\begin{frame}

  \begin{example}
    \begin{itemize}
    \item \LaTeX{}
    \item beamer is easy.
    \end{itemize}    
  \end{example}
\end{frame}


\begin{frame}
  \begin{exampleblock}{Enumerate example:}
    \begin{itemize}
    \item \LaTeX{}
    \item beamer is easy.
    \end{itemize}    
  \end{exampleblock}
\end{frame}


\begin{frame}\frametitle{Examples}
  \emph{Display by column}
  \begin{columns}
    \begin{column}{.5\textwidth}
      \begin{example}
        Itemize example:
        \begin{itemize}
        \item Hello, world! is simple;
        \item \LaTeX{} beamer is easy.
        \end{itemize}
      \end{example}
    \end{column}
    \begin{column}{.5\textwidth}
      \begin{exampleblock}{Enumerate example:}
        \begin{enumerate}
        \item \LaTeX{} beamer is powerful, and
        \item beautiful.
        \end{enumerate}
      \end{exampleblock}
    \end{column}
  \end{columns}
\end{frame}

\begin{frame}\frametitle{Examples}
  \emph{Display by row} \\
  Itemize example: \pause
  \begin{itemize}
  \item Hello, world! is simple;  \pause
  \item \LaTeX{} beamer is easy. \pause
  \end{itemize}
  Enumerate example: \pause
  \begin{enumerate}
  \item \LaTeX{} beamer is powerful, and \pause
  \item beautiful. \pause
  \end{enumerate}
\end{frame}
% \begin{frame}
%   \begin{columns}
%     \begin{column}{.5\textwidth}
%       \begin{exampleblock}{Enumerate example:}
%         \begin{itemize}
%         \item \LaTeX{}
%         \item beamer is easy.
%         \end{itemize}    
%       \end{exampleblock}
%       \begin{column}{.5\textwidth}
%         \begin{exampleblock}{Enumerate example:}
%           \begin{enumerate}
%           \item \LaTeX{} beamer is powerful, and
%           \item beautiful.
%           \end{enumerate}
%         \end{exampleblock}        
%       \end{column}
%     \end{column}
%   \end{columns}
% \end{frame}
\end{document}
%%% Local Variables:
%%% mode: latex
%%% TeX-master: t
%%% End:









