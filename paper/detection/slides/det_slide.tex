\documentclass{beamer}
%\usepackage[UTF8, heading = false, scheme = plain]{ctex}
% \usetheme{Warsaw}
% \usetheme{CambridgeUS}
% \usetheme{Berkeley}
% \usetheme{Antibes}
% \usetheme{Madrid}

%\usetheme{Boadilla}

\begin{document}

\title{Object Detection}
\author{Peng Zhou}
\maketitle{}


\begin{frame}
  \frametitle{}
  SSD Network Architecture
  \begin{figure}[!htb]
    \centering
    \includegraphics[width=1\textwidth]{./pic/ssd-network-architecture.jpg}
  \end{figure}
\end{frame}

\begin{frame}
  \frametitle{}
  Data Augmentation
  \begin{figure}[!htb]
    \centering
    \includegraphics[width=1\textwidth]{./pic/ssd-expansion-sampled.jpg}
  \end{figure}
\end{frame}

\begin{frame}
  \frametitle{}
  Results on VOC 2007 test
  \begin{figure}[!htb]
    \centering
    \includegraphics[width=1\textwidth]{./pic/ssd-results-on-voc-2007.jpg}
  \end{figure}
\end{frame}


\begin{frame}
  \frametitle{Multi-Scale Context Aggregation by Dilated Convolution,ICLR 2016}
\begin{figure}[!htb]
  \centering
  \includegraphics[width=\textwidth]{./pic/dilated-conv.jpg}
\end{figure}
\end{frame}

\begin{frame}
  \frametitle{}
  \begin{figure}[!htb]
    \centering
    \includegraphics[width=\textwidth]{./pic/dilated-conv-context-network.jpg}
  \end{figure}
\end{frame}

\begin{frame}
  \frametitle{}
  \begin{figure}[!htb]
    \centering
    \includegraphics[width=\textwidth]{./pic/dilated-conv-context-module-test.jpg}
  \end{figure}
\end{frame}



\begin{frame}
  \frametitle{}
End
\end{frame}


\begin{frame}
  \frametitle{YOLO9000: Better, Faster, Stronger}
  \begin{itemize}
  \item Multi-Scale Training
    \begin{itemize}
    \item Instead of fixing the input image size we change the net- work every few
      iterations. Every 10 batches our network randomly chooses a new image
      dimension size.
    \end{itemize}
  \end{itemize}
  \begin{figure}[!htb]
    \centering
    \includegraphics[width=0.68\textwidth]{./pic/yolo-tradeoff.jpg}
  \end{figure}
\end{frame}

\begin{frame}

  %\usepackage{subcaption}
  \begin{figure}[ht]
    \centering
    \begin{minipage}{.6\textwidth}
      \centering
      \includegraphics[width=.9\textwidth]{./pic/yolov2-darknet19.jpg}
    \end{minipage}%
    \begin{minipage}{.4\textwidth}
      \begin{itemize}
      \item Similar to the VGG models we use mostly $3\times3$ filters and
        double the number of channels after every pooling step
      \item use global average pooling to make predictions as well as $1\times1$
        filters to compress the feature representation between $3\times3$
        convolutions
      \end{itemize}
    \end{minipage}
  \end{figure}


\end{frame}


\begin{frame}
  \frametitle{YOLO9000: Better, Faster, Stronger}
  \begin{itemize}
  \item Convolutional With Anchor Boxes
    \begin{itemize}
    \item remove the fully connected layers from YOLO and use anchor boxes to
      predict bounding boxes.
    \item Using anchor boxes we get a small decrease in accuracy. YOLO only
      predicts 98 boxes per image but with anchor boxes our model predicts
      more than a thousand.
    \end{itemize}

  \end{itemize}

  \begin{figure}[!htb]
    \centering
    \includegraphics[width=0.8\textwidth]{./pic/yolo-architecture.jpg}
  \end{figure}
  
\end{frame}

\begin{frame}

  \begin{itemize}
  \item Batch Normalization
    \begin{itemize}
    \item Batch normalization leads to sig- nificant improvements in convergence 
    \item By adding batch normalization on all of the convolutional layers in
      YOLO we get more than 2\% improvement in mAP. 
    \end{itemize}
  \end{itemize}
\end{frame}

\begin{frame}

  \begin{figure}[!htb]
    \centering
    \includegraphics[width=\textwidth]{./pic/yolov2-trick.jpg}
  \end{figure}
\end{frame}


\begin{frame}
  \frametitle{YOLO9000: Better, Faster, Stronger}
  
  Closing the dataset size gap between detection and classification.
  \begin{itemize}
  \item Jointly training on classification and detection data
  \item During training we mix images from both detection and classification datasets
    \begin{itemize}
    \item Use WordTree to combine multiple datasets together
    \end{itemize}
  \end{itemize}
  \begin{figure}[!htb]
    \centering
    \includegraphics[height=0.6\textheight]{./pic/yolov2-wordtree.jpg}
  \end{figure}
\end{frame}

\begin{frame}
  \frametitle{YOLO9000: Better, Faster, Stronger}  

  Future work
  \begin{itemize}
  \item For future work we hope to use similar techniques for weakly supervised
    image segmentation.
  \item We also plan to improve our detection results using more powerful matching
    strategies for assigning weak labels to classification data during training.
  \end{itemize}
\end{frame}

\begin{frame}
  \frametitle{Deep residual learning for image recognition. In CVPR 2016}

  %\usepackage{subcaption}
  \begin{figure}[ht]
    \centering
    \begin{minipage}{.5\textwidth}
      \centering
      \includegraphics[width=.9\textwidth]{./pic/cvpr2016-resnet.jpg}
    \end{minipage}%
    \begin{minipage}{.5\textwidth}
      更有力的特征,检测时基于Faster R-CNN的目标检测框架,使用ResNet替换VGG16
      网络可以取得更好的检测结果.
    \end{minipage}
  \end{figure}


\end{frame}

\begin{frame}
  \frametitle{Inside-outside net: Detecting objects in context with skip
    pooling and recurrent neural networks. In CVPR 2016}

  \begin{figure}[!htb]
    \centering
    \includegraphics[width=0.68\textwidth]{./pic/cvpr-inside-outside-network.jpg}
  \end{figure}
  \begin{itemize}
  \item 在Fast R-CNN的基础之上增加context信息,所谓context在目标检测领域是指感兴趣的ROI周围的信息,可以是局部的,也可以是全局的
  \item skip-connection,通过将deep ConvNet的多层ROI特征进行提取和融合,利用该特征进行每一个位置的分类和进一步回归,这也就是inside-network

  \end{itemize}

\end{frame}

\begin{frame}
  \frametitle{HyperNet: Towards Accurate Region Proposal Generation and Joint
    Object Detection. In CVPR 2016}

  \begin{figure}[!htb]
    \centering
    \includegraphics[width=0.8\textwidth]{./pic/cvpr2016-hypernet.jpg}
  \end{figure}

  \begin{itemize}
  \item  神经网络的高层信息体现了更强的语义信息,对于识别问题较为有效;而低层的特征由于分辨率较高,对于目标定位有天然的优势
  \item 作者的贡献点在于如何将deep ConvNet的高低层特征进行融合,进而利用融合后的特征进行region proposal提取和进一步目标检测。

  \end{itemize}
  

\end{frame}


% \begin{frame}
%   \frametitle{The Traditional Method}
%   \begin{enumerate}

%   \item Area selection
%     \begin{itemize}
%     \item The sliding window strategy is used to traverse the whole image. 
%     \item The sliding windows have different scale, different aspect ratio.
%     \item Too many redundant windows, high time complexity
%     \end{itemize}

%   \item Feature extraction
%     \begin{itemize}
%     \item SIFT, HOG
%     \end{itemize}

%   \item Classification
%     \begin{itemize}
%     \item SVM, Adaboost
%     \end{itemize}
%   \end{enumerate}
% \end{frame}

% \begin{frame}
%   \frametitle{The Traditional Method}
%   \emph{Two problems}
%   \begin{enumerate}
%   \item Sliding window selection is time complexity and redundant.
%   \item The features of manual design are not very robust to the variation of diversity.
%   \end{enumerate}
% \end{frame}

% \begin{frame}
%   \frametitle{Deep Learning Method}
% \begin{block}{}
%   \begin{enumerate}
%   \item Deep Learning Object Detection Algorithm Based on Region Proposal.
%   \item Deep Learning Object Detection Algorithm Based on Regression.
%   \item ...
%   \end{enumerate}
% \end{block}
% \end{frame}

% \begin{frame}
%   \frametitle{DL based on Region Proposal}
%   \href{https://people.eecs.berkeley.edu/~rbg/papers/pami/rcnn_pami.pdf}
%   {Region-based Convolution Networks for Accurate Object detection and Segmentation}
%   \begin{block}{Traditional}
%     \begin{itemize}
%     \item Sliding window + features of manual design 
%     \end{itemize}
%   \end{block}
%   \begin{block}{R-CNN}
%     \begin{itemize}
%     \item Region proposal + CNN
%     \end{itemize}
%   \end{block}
%   \begin{figure}[!t]
%     \centering
%     \includegraphics[width=0.7\textwidth]{pic/rcnn1.png}
%     \caption{R-CNN}
%   \end{figure}

% \end{frame}

% \begin{frame}
%   \frametitle{R-CNN}
  
  
% \end{frame}




% \begin{frame}
%   Add items
%   \begin{itemize}
%   \item \LaTeX{}
%   \item beamer is easy.
%   \end{itemize}

%   \begin{enumerate}
%   \item \LaTeX{}
%   \item beamer is powerful
%   \end{enumerate}
% \end{frame}

% \begin{frame}

%   \begin{example}
%     \begin{itemize}
%     \item \LaTeX{}
%     \item beamer is easy.
%     \end{itemize}    
%   \end{example}
% \end{frame}


% \begin{frame}
%   \begin{exampleblock}{Enumerate example:}
%     \begin{itemize}
%     \item \LaTeX{}
%     \item beamer is easy.
%     \end{itemize}    
%   \end{exampleblock}
% \end{frame}


% \begin{frame}\frametitle{Examples}
%   \emph{Display by column}
%   \begin{columns}
%     \begin{column}{.5\textwidth}
%       \begin{example}
%         Itemize example:
%         \begin{itemize}
%         \item Hello, world! is simple;
%         \item \LaTeX{} beamer is easy.
%         \end{itemize}
%       \end{example}
%     \end{column}
%     \begin{column}{.5\textwidth}
%       \begin{exampleblock}{Enumerate example:}
%         \begin{enumerate}
%         \item \LaTeX{} beamer is powerful, and
%         \item beautiful.
%         \end{enumerate}
%       \end{exampleblock}
%     \end{column}
%   \end{columns}
% \end{frame}

% \begin{frame}\frametitle{Examples}
%   \emph{Display by row} \\
%   Itemize example: \pause
%   \begin{itemize}
%   \item Hello, world! is simple;  \pause
%   \item \LaTeX{} beamer is easy. \pause
%   \end{itemize}
%   Enumerate example: \pause
%   \begin{enumerate}
%   \item \LaTeX{} beamer is powerful, and \pause
%   \item beautiful. \pause
%   \end{enumerate}
% \end{frame}
% \begin{frame}
%   \begin{columns}
%     \begin{column}{.5\textwidth}
%       \begin{exampleblock}{Enumerate example:}
%         \begin{itemize}
%         \item \LaTeX{}
%         \item beamer is easy.
%         \end{itemize}    
%       \end{exampleblock}
%       \begin{column}{.5\textwidth}
%         \begin{exampleblock}{Enumerate example:}
%           \begin{enumerate}
%           \item \LaTeX{} beamer is powerful, and
%           \item beautiful.
%           \end{enumerate}
%         \end{exampleblock}        
%       \end{column}
%     \end{column}
%   \end{columns}
% \end{frame}
\end{document}
%%% Local Variables:
%%% mode: latex
%%% TeX-master: t
%%% End:









