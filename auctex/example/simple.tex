\documentclass{article}


\begin{document}
\title{How to structure a \LaTeX{} document}
\author{Peng\\
\emph{zhoupengcv@sjtu.edu.cn}}
\date{\today}
\maketitle{}
\begin{abstract}

This is a simple math exmaple: $$c^2=a^2+b^2$$
\begin{equation}
  \label{eq:1}
  c^2=a^{10} + b^{10}\\
  a^2=c^2-b^2
\end{equation}

\begin{eqnarray}
  \label{eq:2}
  c^2=a^{10} + b^{10}\nonumber\\
  a^2=c^2-b^2
\end{eqnarray}

it will be aligned by the \& \& \\
The * makes it not have an equation number
\begin{eqnarray*}
    x &=& blah blah blah \\ 
      & & more blah blah blah \\
      & & even more blah blah
\end{eqnarray*}

matrix
\begin{equation}       %开始数学环境
\left(                 %左括号
  \begin{array}{ccc}   %该矩阵一共3列,每一列都居中放置
    a11 & a12 & a13\\  %第一行元素
    a21 & a22 & a23\\  %第二行元素
  \end{array}
\right)                 %右括号
\end{equation}

   In this article, I shall discuss some of the fundamental topics in
 producing a structured document.  This document itself does not go into
 much depth, but is instead the output of an example of how to implement
 structure. Its \LaTeX{} source, when in used with my tutorial
 provides all the relevant information.
\end{abstract}

\section{Introduction}
\label{sec:introduction}

This small document is designed to illustrate how easy it is to create a well structured
document within \LaTeX\cite{lamport94}.  You should quickly be able to see how the article
looks very professional, despite the content being far from academic.  Titles, section
headings, justified text, text formatting etc., is all there, and you would be surprised
when you see just how little markup was required to get this output.

\section{Structure}
\label{sec:structure}

One of the great advantages of \LaTeX{} is that all it needs to know is
the structure of a document, and then it will take care of the layout
and presentation itself.  So, here we shall begin looking at how exactly
you tell \LaTeX{} what it needs to know about your document.

\subsection{Top Matter}
\label{sec:top-matter}

The first thing you normally have is a title of the document, as well as
information about the author and date of publication.  In \LaTeX{} terms,
this is all generally referred to as \emph{top matter}.
\begin{itemize}
\item \verb|\title{}| --- The title of the article.
\item \verb|\date| --- The date. Use:
  \begin{itemize}
  \item \verb|\date{\today}| --- to get the date that the document is typeset.
  \item \verb|\date{}| --- for no date
  \end{itemize}
\end{itemize}
\subsection{Author Information}
\label{sec:author-information}

The basic article class only provides the one command:
\begin{itemize}
\item \verb|\author{}| --- The author of the document.
\end{itemize}
It is common to not only include the author name, but to insert new lines (\verb|\\|)
after and add things such as address and email details.  For a slightly more logical
approach, use the AMS article class (\emph{amsart}) and you have the following extra
commands:

\subsection{Sectioning Commands}
\label{sec:sectioning-commands}

The commands for inserting sections are fairly intuitive.  Of course,
certain commands are appropriate to different document classes.
For example, a book has chapters but a article doesn't.
\begin{center}
  \begin{tabular}{|c|c|}
    \hline
    Command & Level \\ \hline
    \verb|\part{}| & -1 \\
    \verb|\chapter{}| & 0 \\
    \verb|\section{}| & 1 \\
    \verb|\subsection{}| & 2 \\
    \verb|\subsubsection{}| & 3 \\
    \verb|\paragraph{}| & 4 \\
    \verb|\subparagraph{}| & 5 \\
    \hline
  \end{tabular}
\end{center}

\begin{thebibliography}{99}
\bibitem{lamport94}
  Leslie Lamport,
  \emph{\LaTeX: A Document Preparation System}.
  Addison Wesley, Massachusetts,
  2nd Edition,1994.
\bibitem{wikibooks}
  http://en.wikibooks.org
\end{thebibliography}


\end{document}
